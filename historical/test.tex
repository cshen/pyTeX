\documentclass[a4]{article}
\usepackage{test}
\usepackage{python}


%-----------------------------------------------------------------------
%                       Début du document
%-----------------------------------------------------------------------
\begin{document}

% Les Bases
Il est possible d'utiliser la langage de script \href{http://www.python.org/}{Python} pour réaliser des boucles, voire des calculs à l'intérieur d'un document \LaTeX~ à l'aide du package \href{http://www.imada.sdu.dk/~ehmsen/pythonlatex.php}{python.sty}\\

\exop{5}\\
Voici le premier énoncé, vous devez compléter la liste des carrés :
\begin{itemize}
\begin{python}
# -*- coding: utf-8 -*-
#!/usr/bin/python

for i in range(10):
    out = r"\item Le carr\'e de %s est %s car $%s \times %s=%s$"%(i, i*i,i,i,i*i)
    print out
print r"\item etc."
\end{python}
\end{itemize}

\par
\vskip 0.5cm

% Développements
Enfin, nous pouvons aussi utiliser le magnifique travail réalisé par l'équipe de \href{http://code.google.com/p/sympy/}{SymPy}, 
nous allons l'utiliser ici pour développer quelques expressions :\\

\exop{5}\\
Voici quelques développements :
\begin{itemize}
\begin{python}
# -*- coding: utf-8 -*-
#!/usr/bin/python

from sympy import latex
from sympy import *

x,y = symbols('xy')

def is_eqal(exp1, exp2):
    exp1 = exp1.replace("$","")
    exp2 = exp2.replace("$","")
    return "$%s=%s$"%(exp1,exp2)

for el in ( (x+y)**2, (x+y)**3, (x+y)**4 ):
    print r"\item[$\bullet$] " + is_eqal(  latex(el), latex(el.expand()) )   

\end{python}
\end{itemize}
\par
\vskip 0.5cm

De même, nous pouvons aussi dériver facilement une expression, la plupart des fonctions usuelles 
étant implémentées dans \href{http://code.google.com/p/sympy/}{SymPy}.\\
% Dérivations
\exop{5}\\
Voici quelques dérivations :
\begin{itemize}

\begin{python}
# -*- coding: utf-8 -*-
#!/usr/bin/python

from sympy import latex
from sympy import *

x,y = symbols('xy')

def is_eqal(exp1, exp2):
    exp1 = exp1.replace("$","")
    exp2 = exp2.replace("$","")
    return "$%s=%s$"%(exp1,exp2)

for el in ( (x+y)**2, tan(x), sin(2*x) ):
    print r"\item[$\bullet$] La d\'eriv\'ee de \textbf{%s} par rapport \`a $x$ est \textbf{%s}"%( latex(el), latex(diff(el,x)) )   

\end{python}
\end{itemize}

\end{document}